\documentclass[11pt]{article}

\textwidth 17cm
\textheight 23cm
\oddsidemargin 0.25cm
\addtolength{\voffset}{-2.4cm}
\addtolength{\hoffset}{-0.5cm}

\setlength{\parindent}{12pt}
\setlength{\parskip}{3pt}
\usepackage{float}

%\usepackage{refcheck}

%\usepackage[caption = false]{subfig}

\usepackage{amssymb,amsmath,epsfig}

%\usepackage[colorlinks=true,breaklinks=true,linkcolor=black,citecolor=black]{hyperref}

\usepackage{graphics,psfrag,graphicx,color,float,subcaption} %amsthm

\numberwithin{equation}{section}
\numberwithin{figure}{section}

\usepackage[table]{xcolor}

% OUR DEFINITIONS %%%%%%%%%%%%%%%%%%%%%%%%%%%%%%%%%%%
\usepackage{sebastyle}
\DeclareMathAlphabet\mathbfcal{OMS}{cmsy}{b}{n}

% END OF OUR DEFINITIONS %%%%%%%%%%%%%%%%%%%%%%%%%%%%%

%\allowdisplaybreaks

%***********************************************************************************

\title{A machine learning method for \\ data driven approaches to sustainability
\thanks{This research was 
partially supported by  Pacific Institute for the Mathematical Sciences, Mitacs, QuanSight, West Grid and Cybera.}}

\author{{\sc Edgar Pacheco}\thanks{Department of ...
email: {\tt your email}.}
\quad
{\sc Symon Islam}\thanks{Department of ...
email: {\tt your email}.}
\quad
{\sc Symon Islam}\thanks{Department of ...
email: {\tt your email}.}
\\
{\sc Thomas Pender}\thanks{Department of ...
email: {\tt your email}.}
\quad
{\sc Igor Pinheiro}\thanks{Department of ...
email: {\tt your email}.}
\quad
{\sc Sebasti\'an Moraga}\thanks{ Department of Mathematics, Simon Fraser University. Canada.
email: {\tt smoragas@sfu.ca}.}}

\date{ }

\begin{document}

\maketitle
\MG{This is jus  the first draft, everything should be attained to changes. Please feel free to edit this document.}
\begin{abstract}
\noindent
In this paper we propose and analyze [...] In this part we write the possible abstract for the project.

\end{abstract}

\noindent
{\bf Key words}: Machine learning, climate change, optimization, ...

\smallskip\noindent

                                                  
%%%%%%%%%%%%%%%%%%%%%%%%%%%%%%%%%%%%%%%%%%%%%%%%%%%%%%%%%%%%%%%%%%

\section{Introduction}\label{section1}

Sustainability is the greatest challenge facing the human race. The United Nations (U.N.) Sustainable Development Goals describe a multi-faceted view of sustainability covering environment, economic and societal factors (see, \ref{}).
Agriculture is one of the most significant areas of economic activity for countries around the world and has significant amount of environmental impact. It is estimated that agriculture accounts for at least $10\%$ of green house gas emissions in many countries (see, \ref{}). In many areas the impact of agriculture is seen through changing land use, increased emissions and increased water consumption, leading to lasting changes to the rest of the environment.
This challenge is focused on the use of open data to improve understanding and ideally predictability for environmental impact from agriculture. In nearly all countries there is impact from agriculture through changing land use, water consumption, use of fertilizer, GHG and other emissions. From a local perspective agriculture creates different dynamics for indigenous plant and animal life in addition to creating different micro-climates. On a more broad scale, agriculture can put pressure on entire river systems and lead to changing weather patterns as atmospheric humidity and solar radiation emission changes for land and sea.
There are many data sources available to investigate environmental changes and impacts due to agriculture. Similarly there are an increase number of data sets which convey information about practices, agriculture production and green house gas emissions. Combining data across data sources and interpreting data in new ways could provide better insights on sustainability. Creating models to describe impact which can be predictive using machine learning could improve planning and shift practices to reflect longer term environmental impact. Utilizing technology like Blockchain may provide a foundation to create an immutable data ledger for environmental impact while also leveraging smart contracts to take action on data when conditions are met.
In the ideal outcome, design of a model and a system of action for environmental impact which leverages open data provides a complement to the current TheoryMesh system which is capturing individual operation level impact from farm activities.

\subsection*{Outline}
\MG{Sebastian: Here a little bit of outline of the  project, we  should write this aprt after the main parts are in place.}

\section{Motivation}
\begin{itemize}
\item Bullet points for the main motivation.
\item Should talk about the importance of the model
\item Why does it fit the data.

\item etc.
\end{itemize}
\section{Contributions}
\begin{itemize}
\item Contributions of our work
\item How does our work  (try to-) fix the problem. etc
\end{itemize}
\section{Related work}

\begin{itemize}
\item A little of literature review
\end{itemize}
\section{Setup}

\begin{itemize}
\item Main mathematical setup. All that is needed to  understand the problem.
\MG{Sebastian: Maybe this part changes as we  go  with the project, because should be something more for the industry  than theoretical.}
\end{itemize}
\section{The model problem}

\begin{itemize}
\item What kind of model, prediction are we using. Machine learning approaches, regression models, etc...
\end{itemize}
\section{Problem statement}

\begin{itemize}
\item We should talk a bit of the  statement of the problem 
\end{itemize}
\section{Main results}

\section{Conclusions}

\section{Future work}
%%%%%%%%%%%%%%%%%%%%%%%%%%%%%%%%%%%%%%%%%%%%%%%%%%%%%%%%%%%%%%%%%%%%%%%%%%%%%%%%%%%%%%%%%%

\begin{thebibliography}{99}

\bibitem{Adams}
{\sc R.A. Adams and J.J.F. Fournier},
Sobolev Spaces. Second edition. Pure and Applied Mathematics (Amsterdam), 140. 
Elsevier/Academic Press, Amsterdam, 2003. 

 

\end{thebibliography}

\end{document}

